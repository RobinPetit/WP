% si on ne précise pas le sort explicitement, il n'est pas content... Ça me gonfle donc je laisse comme ça pour l'instant...
% //~ TODO modifier le Sort des 2 derniers selon votre convenance

\newglossaryentry{creature}{
    type=glossary,
    name={creature},
    description={Une créature, (ou \textbf{Creature}) est une carte jouable sur le plateau. Elle dispose d'un certain montant de point de vie, d'un coût d'invocation, d'une valeur d'attaque et éventuellement d'un ou plusieurs effets, qu'il soit direct ou qu'il applique des contraintes à qui que ce soit.},
    sort=0
}

\newglossaryentry{deck}{
    type=glossary,
    name={deck},
    description={Paquet d'exactement 20 cartes de jeu ne comprenant pas plus de deux fois la même carte},
    sort=10
}

\newglossaryentry{duel}{
    type=glossary,
    name={duel},
    description={Affrontement entre deux utilisateurs se solvant en la victoire de l'un des deux joueurs},
    sort=20
}

\newglossaryentry{facultatif}{
    type=glossary,
    name={facultatif},
    description={Non-requis explicitement ou implicitement par le client mais qui pourra être implémenté par la suite},
    sort=30
}

\newglossaryentry{fosse}{
    type=glossary,
    name={fosse},
    description={\textit{Pseudo-lieu} du plateau où sont stockées les cartes ayant servi et étant arrivées en fin de vie},
    sort=40
}

\newglossaryentry{invoquer}{
    type=glossary,
    name={invoquer},
    description={Utiliser une carte. Consiste à placer un monstre sur le terrain, attquer avec un monstre ou lancer un sort},
    sort=50
}

\newglossaryentry{lobby}{
    type=glossary,
    name={lobby},
    description={\textit{Pseudo} salle d'attente où les joueurs patientent le temps que le serveur leur désigne un adversaire},
    sort=60
}

\newglossaryentry{mutex}{
    type=glossary,
    name={mutex},
    description={En informatique, un mutex permet de régler les soucis d'accès simultanées à une même donnée.},
    sort=70
}

\newglossaryentry{sort}{
    type=glossary,
    name={sort},
    description={Un sort, (ou \textbf{Spell}) est une carte qui déclenche un effet à l'activation. Un sort ne dipose que d'un coût, et d'un ou plusieurs effets.},
    sort=80
}

\newglossaryentry{pop}{
    type=glossary,
    name={pop},
    description={Un pop est une action courante sur les piles informatique. Cette action consiste à enelever le sommet de la pile, et à souvent l'utiliser par la même occasion},
    sort=90
}
\newglossaryentry{push}{
    type=glossary,
    name={push},
    description={Un push est une action courante sur les piles informatique. Cette action consiste à ajouter un élément au dessus de la pile.},
    sort=100
}

