\documentclass{article}

\usepackage[utf8]{inputenc}
\usepackage[T1]{fontenc}
\usepackage{times}
\usepackage[cm]{fullpage}
\usepackage[parfill]{parskip}
\usepackage[xindy]{glossaries}
\usepackage{glossary-mcols}
\usepackage[french]{babel}
\usepackage{lmodern}
\usepackage{graphicx}

\graphicspath{{"Diagrammes UML/"}}

\title{Software Requirement Document : Wizard Poker}
\author{Baudoux Nicolas \and Berrewaerts Jonathan \and Gueniffrey Stanislas \and Muranovic Allan \and Petit Robin \and Reynouard Alexis \and Verhelst Théo}
\date{année académique 2015-2016}

\newglossary[glg]{glossary}{slm}{sbl}{Glossaire}
\newglossaryentry{groupe}
{
    name={Groupe},
    description={Explication d'un groupe}
}

\makeglossaries

\newglossarystyle{noIndex} {
    \renewenvironment{theglossary}
    {\vspace{-4em} 
    \begin{longtable}{@{}p{0.15\textwidth}@{}p{0.85\textwidth}}}
    {\end{longtable}}
    \renewcommand*{\glossaryheader}{}
    \renewcommand*{\glsgroupheading}[1]{}
    \renewcommand*{\glossaryentryfield}[5]{\textbf{\glstarget{##1}{##2}} & ##3. \\}
}

\newglossarystyle{Index} {
    \renewenvironment{theglossary}
    {\vspace{-4em} 
    \begin{longtable}{@{}p{0.15\textwidth}@{}p{0.85\textwidth}}}
    {\end{longtable}}
    \renewcommand*{\glossaryheader}{}
    \renewcommand*{\glsgroupheading}[1]{}
    \renewcommand*{\glossaryentryfield}[5]{\textbf{\glstarget{##1}{##2}} & ##5. \\}
}

\begin{document}

\pagenumbering{Roman}
\maketitle
\tableofcontents
\newpage
\pagenumbering{arabic}

\section{Introduction}
    \subsection{But du projet}
        L'objectif visé par ce projet d'année de BA2 en informatique est la réalisation d'une application client-serveur en C/C++. L'application visée est
        un jeu de carte (appelé \textit{Wizard Poker}) tour à tour et multijoueurs en réseau. Elle est destinée à tout type de public, \textit{open source}
        libre de droit et à but non commercial. C'est un projet académique.  % licence à définir.

        Pour ce faire, notre équipe, composée de sept personnes, dispose de trois phases de développement qui dureront approximativement 4 semaines chacune :
        \begin{itemize}
            \item la première portera uniquement sur la création du squelette respectant toutes les demandes effectuées par les client ;
            \item la deuxième phase concernera l'implémentation en console uniquement ;
            \item et enfin la troisième ajoutera l'interface graphique.
        \end{itemize}

        \subsubsection{Description du Wizard Poker}
            Le Wizard Poker est un jeu de cartes dans lequel deux joueurs s'affrontent en \gls{duel} au tour par tour. Chaque joueur a son propre \gls{deck}
            choisi parmi sa collection de cartes. Au début de chaque duel, les deux joueurs ont le même nombre de points de vie (à savoir 20) et piochent 5
            cartes de leur \gls{deck}. Chacun à leur tour, les joueurs invoquent des cartes à l'aide des points d'énergie qu'ils reçoivent à chaque début de tour.
            Les cartes peuvent soit attaquer l'adversaire (directement ses points de vie ou alors les cartes invoquées par l'adversaire), soit avoir
            un effet \textit{spécial}\footnote{Par exemple : invoquer une carte au hasard, redonner de la vie, augmenter l'attaque, etc.}. Lorsqu'une carte
            arrive à la fin de sa période d'existence (points de vie de la carte arrivés à 0, sort détruisant ladite carte, effet terminé, etc.), elle est
            défaussée, c'est-à-dire qu'elle est envoyée dans la \gls{fosse}. Lorsqu'une carte est dans la fosse (également appelé cimetière), elle
            peut éventuellement être réutilisée à l'aide de sorts particuliers par exemple. Une carte défaussée n'est pas perdue, elle reste propriété du joueur :
            à la fin de la partie, les joueurs ont les mêmes cartes qu'au début de la partie si ce n'est que le gagnant a reçu une carte supplémentaire
            suite à sa victoire. La partie s'arrête lorsqu'un des deux joueurs a un nombre inférieur ou égal à 0 points de vie ou a passé 10 tours sans carte en main.

    \subsection{Glossaire}  % doit s'agrandir avec le temps (automatiquement)
        \printglossary[type=glossary, style=noIndex, title=]

    \subsection{Historique}
        \begin{itemize}
            \item[11/12/2015] réunion et squelette du SRD (équipe) ;
            \item[15/12/2015] création des diagrammes UML (équipe UML) ;  % à expliciter !!
            \item[15/12/2015] rédaction de la version propre du SRD (?).
        \end{itemize}

\section{Besoins de l'utilisateur}
    L'utilisateur de l'application, à savoir le joueur, a un certain nombre de besoins. Ces derniers doivent être satisfaits afin de garantir un confort d'utilisation
    maximal. En utilisant l'application, le joueur \textbf{doit} avoir la possibilité de :

    \begin{itemize}
        \item créer un compte utilisateur ;
        \item créer un \gls{deck} ou en modifier un pré-existant ;
        \item consulter les cartes dont il dispose ;
        \item consulter les \glspl{deck} dont il dispose ;
        \item ajouter un joueur existant en ami ;
        \item discuter avec un ami ;
        \item \textbf{\gls{facultatif} :} discuter avec plusieurs amis simultanément ;
        \item affronter un adversaire aléatoire ;
        \item défier un joueur de sa liste d'amis ;
        \item consulter le classement des joueurs.
    \end{itemize}

    \subsection{Exigences fonctionnelles}
        \subsubsection{Possibilités d'actions de l'utilisateur}
            \begin{center}\includegraphics[scale=0.6]{UseCase1_Main.png}\end{center}

        \subsubsection{Création d'un compte}
            \begin{center}\includegraphics[scale=0.6]{Collaboration1_SignIn.png}\end{center}

            \begin{description}
                \item[Description] L'utilisateur peut créer un compte à l'aide d'un identifiant unique (son pseudo) et d'un mot de passe associé ;
                \item[Pré-condition] l'utilisateur doit être connecté au serveur ;
                \item[Post-condition] Soit la demande est acceptée (identifiant admissible), soit la demande est rejetée (identifiant non-admissible ou déjà utilisé).
            \end{description}

        \subsubsection{Gestion des \glspl{deck}}
            \begin{center}\includegraphics[scale=0.5]{Model1_DecksManagement.png}\end{center}

        \subsubsection{Début de partie}
            \begin{center}\includegraphics[scale=0.45]{Collaboration2_StartDuel.png}\end{center}

            \begin{description}
                \item[Description] un joueur connecté peut défier un ami ou demander un duel contre un adversaire aléatoire ;
                \item[Pré-condition] chaque utilisateur doit être identifié et connecté à son compte ;
                \item[Post-condition] un joueur aura perdu, l'autre aura gagné\footnote{Le gagnant reçoit une carte aléatoire à ajouter à son \gls{deck}.}.
            \end{description}

    \subsection{Exigences non-fonctionnelles}
        L'application \textbf{doit} tourner sur Linux (salles 008 et 007 du NO.4) et fonctionner en réseau (\textit{a priori} avec clients
        et serveurs sur des machines séparées).

        De plus, l'utilisateur doit pouvoir discuter par messages à n'importe quel moment de l'exécution de l'application : tant pendant
        qu'il gère ses amis, ses cartes ou pendant qu'il est en \gls{duel}. Ces discutions ne peuvent se faire avec un joueur
        n'étant pas ami de l'utilisateur.

    \subsection{Exigences de domaine}
        Les exigences implicites au domaine du jeu sont les suivante :

        \begin{itemize}
            \item \textbf{\gls{facultatif} : } l'application doit être accessible à tout utilisateur potentiel ;
            \item \textbf{\gls{facultatif} : } l'application doit être \textit{amusante} donc équilibrée ;
            \item \textbf{\gls{facultatif} : } empêcher la triche dans la mesure du possible.
        \end{itemize}

\section{Besoins du système}
    \subsection{Exigences fonctionnelles}
        Dans un premier temps, l'application \textbf{doit} tourner en ligne de commande et doit, dans un second temps adopter une interface
        graphique. De plus, l'application doit être développée en C++.

    \subsection{Exigences non-fonctionnelles}
        Le système \textbf{doit} être maintenable et pensé dans l'optique d'une future adaptation avec interface graphique.

    \subsection{Design et fonctionnement du système}
        La structure du programme est résumée dans les classes suivantes (servant également de \textit{pseudo-squelette de code}) :

        \begin{center}\includegraphics[scale=0.5]{Model2_Classes.png}\end{center}

\section{Index}
        \printglossary[type=glossary, style=Index, title=]

\end{document}
