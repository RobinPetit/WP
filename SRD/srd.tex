\documentclass[12pt]{article} 

\usepackage[utf8]{inputenc}
\usepackage[T1]{fontenc}
\usepackage[cm]{fullpage}
\usepackage[parfill]{parskip}
\usepackage[bottom]{footmisc}  % notes de bas de page réellement en bas de page
\usepackage[francais]{babel} % have to be loaded before glossaries (as many others)

%%%%% fonts %%%%%

%\usepackage{lmodern} % if you don't like baskervad, uncomment this and comment out below
% Baskervad
\usepackage[lf]{Baskervaldx} % lining figures
\usepackage[bigdelims,vvarbb]{newtxmath} % math italic letters from Nimbus Roman
\usepackage[cal=boondoxo]{mathalfa} % mathcal from STIX, unslanted a bit
\renewcommand*\oldstylenums[1]{\textosf{#1}}

%%%%% graphicx %%%%%

\usepackage{graphicx}

\graphicspath{{"Diagrams/"}}

%%%%% glossaries %%%%%

\usepackage[nogroupskip,
            toc, 
			numberedsection,
			style=list]
		{glossaries}
\usepackage{glossary-mcols}

\newglossary[glg]{glossary}{slm}{sbl}{Glossaire}
\newglossaryentry{groupe}
{
    name={Groupe},
    description={Explication d'un groupe}
}

\makenoidxglossaries

\newglossarystyle{listNoIndex}{%
  \renewcommand*{\glossaryentryfield}[5]{%
    \item[\glsentryitem{##1}\glstarget{##1}{##2}]
       ##3\glspostdescription}%
  \renewcommand*{\glossarysubentryfield}[6]{%
    \glssubentryitem{##2}%
    \glstarget{##2}{\strut}##4\glspostdescription}%
}

\newglossarystyle{indexNoDescription}{%
  \renewcommand*{\glossaryentryfield}[5]{%
    \item\glsentryitem{##1}\textbf{\glstarget{##1}{##2}}%
      \ifx\relax##4\relax
      \else
        \space(##4)%
      \fi % jusque là rien d'anormal
	  % ici on n'affiche pas la description
      \space ##5}% puis on affiche la liste des utilisations
  % meme principe ci-dessous
  \renewcommand*{\glossarysubentryfield}[6]{%
    \ifcase##1\relax
      % level 0
      \item
    \or
      % level 1
      \subitem
      \glssubentryitem{##2}%
    \else
      % all other levels
      \subsubitem
    \fi
    \textbf{\glstarget{##2}{##3}}%
    \ifx\relax##5\relax
    \else
      \space(##5)%
    \fi
    \space ##6}%
}%

%%%%% I hate code repetitions in latex %%%%%

\let\simplesection\section
\renewcommand{\section}{\clearpage\simplesection}

\usepackage{needspace}
\let\simplesubsection\subsection
\renewcommand{\subsection}{
	\needspace{0.5\pagetotal}
	\simplesubsection
}

\newcommand{\doit}{\textbf{doit} }
\newcommand{\faclt}{\textbf{\gls{facultatif} : }}

\begin{document}

\pagenumbering{Roman}
\begin{titlepage}

\newcommand{\HRule}{\rule{\linewidth}{0.5mm}} % Defines a new command for the horizontal lines

\center % Center everything on the page

%----------------------------------------------------------------------------------------
%	HEADING SECTIONS
%----------------------------------------------------------------------------------------

\textsc{\LARGE Université libre de Bruxelles}\\[1.5cm]
\textsc{\Large INFO-F-209}\\[0.5cm]
\textsc{\large Projet d'année 2}\\[0.5cm]

%----------------------------------------------------------------------------------------
%	TITLE SECTION
%----------------------------------------------------------------------------------------

\HRule \\[0.4cm]
{ \huge \bfseries Software Requirement Document : Wizard Poker \\ Groupe 5}\\[0.4cm]
\HRule \\[1.5cm]

%----------------------------------------------------------------------------------------
%	AUTHOR SECTION
%----------------------------------------------------------------------------------------

\Large \emph{Auteurs :}\\
Baudoux Nicolas, Gueniffey Stanislas,  Muranovic Allan, Petit Robin, Reynouard Alexis et Verhelst Théo

%----------------------------------------------------------------------------------------
%	DATE SECTION
%----------------------------------------------------------------------------------------

{\large \today}\\[2cm]

%----------------------------------------------------------------------------------------
%	LOGO SECTION
%----------------------------------------------------------------------------------------

\includegraphics{ulb.png}\\[1cm]

%----------------------------------------------------------------------------------------

\vfill

\end{titlepage}

\tableofcontents

\pagenumbering{arabic}
\section{Introduction}
	\subsection{But du projet}
		L'objectif visé par ce projet d'année de BA2 en sciences informatiques est la réalisation d'une application client-serveur en C/C++. L'application visée est
		un jeu de carte (appelé \textit{Wizard Poker}) au tour par tour et multijoueur en réseau. Elle est destinée à tous types de public. Elle est également
		\textit{open source}, libre de droit et à but non commercial : c'est un projet académique.

		Pour ce faire, notre équipe, composée de sept personnes, dispose de trois phases de développement qui dureront approximativement 4 semaines chacune :
		\begin{itemize}
			\item la première portera uniquement sur la création du squelette respectant toutes les demandes effectuées par les client ;
			\item la deuxième phase concernera l'implémentation en console uniquement ;
			\item et enfin la troisième ajoutera l'interface graphique.
		\end{itemize}

		\subsubsection{Description du Wizard Poker}
			Le Wizard Poker est un jeu de cartes dans lequel deux joueurs s'affrontent en \gls{duel} au tour par tour.
			
			Chaque joueur a son propre \gls{deck} choisi parmi sa collection de cartes. Au début de chaque \gls{duel}, les deux joueurs ont le même nombre de points de vie (à savoir 20) et piochent 5 cartes de leur \gls{deck}.
			Le premier joueur sera choisi aléatoirement par le serveur. Ensuite, tour à tour, les joueurs pourront utiliser leurs cartes, qu'elles soient un \gls{sort} ou une \gls{creature} tant que leur coût d'utilisation/d'invocation ne dépasse pas l'énergie disponible du joueur (le joueur gagne un point d'énergie supplémentaire par tour jusqu'à atteindre son maximum de 10).
			
			Les cartes peuvent attaquer soit l'adversaire directmeent, soit une \gls{creature} adverse. Ou encore avoir un ou plusieurs effet(s) \textit{spéciaux}\footnote{Par exemple : \gls{invoquer} une carte au hasard, redonner de la vie, augmenter l'attaque, etc.}. 
			Lorsqu'une carte arrive à la fin de sa période d'existence (points de vie de la carte arrivés à 0, \gls{sort} détruisant ladite carte, effet terminé, etc.), elle est défaussée, c'est-à-dire qu'elle est envoyée dans la \gls{fosse}.
			
			Lorsqu'une carte est dans la \gls{fosse} (également appelé cimetière), elle peut éventuellement être réutilisée à l'aide de \glspl{sort} particuliers par exemple. Une carte défaussée n'est pas perdue, elle reste propriété du joueur : à la fin de la partie, les joueurs ont les mêmes cartes qu'au début de la partie si ce n'est que le gagnant a reçu une carte supplémentaire suite à sa victoire. La partie s'arrête lorsqu'un des deux joueurs a un nombre inférieur ou égal à 0 points de vie ou a passé 10 tours sans carte en main.

			\begin{center}\includegraphics[scale=0.45]{Activity1_Gameplay.png}\end{center}

	%\subsection{Glossaire}  
		%\printglossary[type=glossary, style=noIndex, title=]
	\setglossarysection{subsection}
	\printnoidxglossary[title=Glossaire,style=listNoIndex] % doit s'agrandir avec le temps (automatiquement)

	\subsection{Historique}
		\begin{itemize}
			\item[11/12/2015] Réunion (équipe complète) ;
			\item[11/12/2015] squelette du SRD (équipe SRD) ;
			\item[15/12/2015] création des diagrammes UML (équipe UML) ;
			\item[15/12/2015] rédaction de la version propre du SRD (Robin) ;
			\item[31/01/2016 $\rightarrow$ 26/02/2016] développement du projet (en console uniquement).
			\item[07/02/2016] réunion (équipe complète) ;
			\item[11/02/2016] réunion (équipe complète) ;
			\item[15/02/2016] réunion (équipe complète) ;
			\item[22/02/2016] réunion (équipe complète) ;
			\item[26/02/2016] mise à jour du SRD (Allan et Nicolas).
		\end{itemize}



\section{Besoins de l'utilisateur}
	L'utilisateur de l'application, à savoir le joueur, a un certain nombre de besoins. Ces derniers doivent être satisfaits afin de garantir un confort d'utilisation
	maximal. En utilisant l'application, le joueur \doit avoir la possibilité de :

	\begin{itemize}
		\item créer un compte utilisateur ;
		\item consulter, modifier ou supprimer ses informations personnelles ; 
		\item créer un \gls{deck} ou en modifier un pré-existant ;
		\item consulter les cartes dont il dispose ;
		\item consulter les \glspl{deck} dont il dispose ;
		\item ajouter un joueur existant en ami ;
		\item discuter avec un ami ;
		\item \faclt discuter avec plusieurs amis simultanément ;
		\item affronter un adversaire aléatoire ;
		\item \faclt défier un joueur de sa liste d'amis ;
		\item consulter le classement des joueurs.
	\end{itemize}

	\subsection{Exigences fonctionnelles}
		\subsubsection{Possibilités d'actions de l'utilisateur}
			\begin{center}\includegraphics[scale=0.5]{UseCase1_Main.png}\end{center}
			\begin{itemize}
				% \TODO: remplacer Gérer un compte par Créer un compte dans UseCase1_Main.png
				\item \textbf{Créer un compte}
				\begin{description}
					\item[Description] Un utilisateur doit créer un compte pour pouvoir accéder à l'application.
					\item[Pré-condition] Aucune.
					\item[Post-condition] Un compte est créé, le nom d'utilisateur correspondant est unique.\\
				\end{description}

				\item \textbf{Se connecter}
				\begin{description}
					\item[Description] Un utilisateur doit se connecter avec un compte pour pouvoir accéder à l'application.
					\item[Pré-condition] L'utilisateur n'est pas déjà connecté, le compte utilisé est existant et n'est pas déjà connecté.
					\item[Post-condition] L'utilisateur est connecté.\\
				\end{description}

				\item \textbf{Consulter sa collection}
				\begin{description}
					\item[Description] Un utilisateur peut voir la collection de cartes dont il dispose.
					\item[Pré-condition] L'utilisateur doit être connecté au serveur.
					\item[Post-condition] L'utilisateur a pris connaissance de sa collection de cartes.\\
				\end{description}

				\item \textbf{Faire une partie}
				\begin{description}
					\item[Description] Un utilisateur peut jouer une partie.
					\item[Pré-condition] Aucune.
					\item[Post-condition] Si l'utilisateur a gagné la partie, une carte est ajoutée à sa collection.
					Ses statistiques (parties gagnées / parties perdues) sont mises à jour.\\
				\end{description}

				\item \textbf{Discuter avec ses amis}
				\begin{description}
					\item[Description] Un utilisateur peut entamer une conversation avec d'autres joueurs, s'ils sont dans sa liste d'amis.
					\item[Pré-condition] L'utilisateur a au moins un ami, et l'ami est connecté.
					\item[Post-condition] L'utilisateur est en conversation avec son ami.
				\end{description}
			\end{itemize}

		\subsubsection{Création d'un compte}
			\begin{center}\includegraphics[scale=0.6]{Collaboration1_SignIn.png}\end{center}

			\begin{description}
				\item[Description] L'utilisateur peut créer un compte à l'aide d'un identifiant unique (son pseudo) et d'un mot de passe associé.
				\item[Pré-condition] L'utilisateur doit être connecté au serveur.
				\item[Post-condition] Soit la demande est acceptée (identifiant admissible), soit la demande est rejetée (identifiant non-admissible ou déjà utilisé).
			\end{description}

		\subsubsection{Gestion des \glspl{deck}}
			\begin{center}\includegraphics[scale=0.5]{Model1_DecksManagement.png}\end{center}
			\begin{itemize}
				\item \textbf{Supprimer un \gls{deck}}
				\begin{description}
					\item[Description] Suppression d'un \gls{deck}.
					\item[Pré-condition] La liste des \glspl{deck} n'est pas vide.
					\item[Post-condition] Il y a un \gls{deck} en moins dans la liste des \glspl{deck}.\\
				\end{description}

				\item \textbf{Afficher un \gls{deck}}
				\begin{description}
					\item[Description] Affichage d'un \gls{deck}.
					\item[Pré-condition] La liste des \glspl{deck} n'est pas vide.
					\item[Post-condition] L'utilisateur a pris connaissance du contenu d'un \gls{deck}.\\
				\end{description}

				\item \textbf{Créer un \gls{deck} par défaut}
				\begin{description}
					\item[Description] Pour créer un nouveau \gls{deck}, le \gls{deck} par défaut est ajouté.
					\item[Pré-condition] Aucune.
					\item[Post-condition] Il y a un \gls{deck} en plus dans la liste des \gls{deck}.\\
				\end{description}

				\item \textbf{Changer les cartes d'un \gls{deck}}
				\begin{description}
					\item[Description] Pour modifier un \gls{deck}, il faut changer une par une les cartes qui le composent.
					\item[Pré-condition] La liste des \glspl{deck} n'est pas vide.
					\item[Post-condition] Un des \glspl{deck} de la liste a été modifié.\\
				\end{description}
			\end{itemize}
			Le fonctionnement de la création d'un nouveau \gls{deck} nécessite peut-être quelques explications
			supplémentaire. 
			Le use case « Créer un \gls{deck} » commence par ajouter dans la liste des \glspl{deck}
			une copie du \gls{deck} par défaut (celui donné à la création du compte). Ensuite l'utilisateur
			est redirigé vers le use case "Changer les cartes d'un \gls{deck}", qui permet de remplacer
			une par une les cartes d'un \gls{deck}. 
			Ainsi, avec une grande modularité et peu de code,
			il est donné à l'utilisateur la possibilité de créer un tout nouveau \gls{deck} sans devoir recoder une partie
			de l'interface qui modifie un \gls{deck}.

		\subsubsection{Début de partie}
			\begin{center}\includegraphics[scale=0.45]{Collaboration2_StartDuel.png}\end{center}

			\begin{description}
				\item[Description] Un joueur connecté peut défier un ami ou demander un \gls{duel} contre un adversaire aléatoire.
				\item[Pré-condition] Chaque utilisateur doit être identifié et connecté à son compte.
				\item[Post-condition] Un joueur aura perdu, l'autre aura gagné\footnote{Le gagnant reçoit une carte aléatoire à ajouter à sa collection.}.
			\end{description}

		\subsubsection{Détail d'un tour}
			\begin{center}\includegraphics[scale=0.6]{Collaboration3_RoundDetails.png}\end{center}

	\subsection{Exigences non-fonctionnelles}
		L'application \doit tourner sur Linux Debian (salles 008 et 009 du NO.4) et fonctionner en réseau (\textit{a priori} avec clients
		et serveurs sur des machines séparées).

		De plus, l'utilisateur \doit pouvoir discuter par messages à n'importe quel moment de l'exécution de l'application : tant pendant
		qu'il gère ses amis, ses cartes ou pendant qu'il est en \gls{duel}. Ces discussions ne peuvent se faire avec un joueur
		n'étant pas ami de l'utilisateur.

	\subsection{Exigences de domaine}
		Les exigences implicites au domaine du jeu, voir ci-dessous, sont facultatives :

		\begin{itemize}
			\item l'application doit être accessible à tout utilisateur potentiel ;
			\item l'application doit être \textit{amusante} donc équilibrée ;
			\item empêcher la triche dans la mesure du possible.
		\end{itemize}



\section{Besoins du système}
	\subsection{Exigences fonctionnelles}
		Dans un premier temps, l'application \doit tourner en ligne de commande et \doit, dans un second temps adopter une interface
		graphique.
		De plus, l'application \doit être développée en C++.

	\subsection{Exigences non-fonctionnelles}
		Le système \doit être maintenable et pensé dans l'optique d'une future adaptation avec interface graphique.

	\subsection{Design et fonctionnement du système}
		La structure du programme est résumée dans le diagramme de classe ci-dessous. Chaque élément sera développé par après.
		\begin{center}\includegraphics[scale=0.5]{Global.png}\end{center}

		Nous pouvons constater que le client (Player) se connecte au serveur qui, lui, gère les parties (le tableau de jeu) ainsi que la base de donnée et tout ce que cette dernière comprend.


		\subsubsection{Diagramme de classe du \gls{deck} et des cartes}
		\begin{center}\includegraphics[scale=0.5]{deck.png}\end{center}
			Comme indiqué sur ce diagramme, un \gls{deck} est composé de cartes, 20 pour être précis. Les cartes sont soit des créatures
			soit des \glspl{sort} et on ne peut posséder que maximum deux exemplaire d'une même carte au sein d'un même \gls{deck}.
			De plus, les \glspl{deck} sont indépendants, il est donc possible de posséder uniquement deux fois la même carte, mais avoir plusieurs \glspl{deck} la possédant au moins une fois.

			L'application des effets est simple : chaque effet est un effet direct, que ce soit infliger X points de dégats, comme soigner un allié ou encore interdire le joueur adverse de jouer pendant Y tours.
			Un effet plus complexe est tout aussi direct, si ce n'est qu'il va modifier la liste de contraintes de la partie et c'est cette liste qui va permettre d'appliquer des effets sur la durée, ou des effets qui respectent certaines conditions (appliquer tel effet tant que telle créature est en vie par exemple).
	

		\subsubsection{Diagramme de classe du serveur et de sa base de donnée}
		\begin{center}\includegraphics[scale=0.5]{Server.png}\end{center}
			Ici, la structure est un peu plus complexe : le serveur est composé d'une base de donnée en SQL qui est interfacée par une base de donnée utilisable directement en C++ pour l'application serveur.

			C'est donc le serveur qui va naturellement tout gérer : les parties, les requêtes d'amitié en passant,
			la gestion/modification de \glspl{deck}/de comptes, etc. Tout sera stocké de son côté dans la base de données.
			Le client ne possédera aucune donnée, il est obligé de passer par les serveur quelle que soit son intention\footnote{Il existe une base de
			données chez le client afin de stocker les informations n'impactant pas le \textit{déroulement propre du jeu}, à savoir les descriptions des cartes,
			les images (pour la partie graphique), etc.}.



		\subsubsection{Diagramme des classes intervenant dans une partie (simplifié)}
		\begin{center}\includegraphics[scale=0.5]{Game.png}\end{center}
			Il est à noter que c'est un diagramme de classe, et non d'activités, il regroupe juste les plus grosses classes intervenant dans la partie,
			et non pas leur lien/fonctionnement entre elles.
			Pour qu'une partie puisse avoir lieu, il nous faut 2 joueurs connecté au serveur qui s'occupera lui-même de modifier le plateau de jeu.


		\subsubsection{Diagramme des classes nécessaires au menu en console}
			Toutes les métodes servant à une expérience de jeu optimal sont implémentées, mais en console.
			L'interface graphique arrive à la prochaine étape du développement de ce projet, tout ce qui gère le fonctionnement du jeu est néanmoins présent.

			Enfin, voici les différentes classes représentant le menu, le tout est divisé en 2 parties pour plus de clarté.

			\subsubsection{Première moitié : Menu et fonctionnalités principales}
			\begin{center}\includegraphics[scale=0.5]{stack1.png}\end{center}
				C'est au moyen de piles que nous avons implémenté un menu fluide, efficace et facile d'utilisation.

				En effet, pour entrer dans un nouveau menu, il suffit de \textit{\gls{push}} le choix, et pour un retour au menu précédent, il suffit de faire un \textit{\gls{pop}}. 

				Ainsi, pour le menu et toutes ses fonctionnalités, que ce soit au niveau de la gestion des amis jusqu'au menu principal, 
				nous avons une classe abstraite héritée (\textbf{AbstractState}) ainsi que l'utilisation d'une deuxième classe (\textbf{StateStack}). C'est grâce à ces deux dernière que tout 
				implémentation supplémentaire dans le menu ou autre est possible.


			\subsubsection{Deuxième moitié: fonctionnalités plus avancées}
			\begin{center}\includegraphics[scale=0.5]{stack2.png}\end{center}
				Ici, nous nous occupons un peu plus de l'expérience de jeu. Nous avons un gestionnaire de cartes, un gestionnaire de \gls{deck} ainsi qu'un classement des joueurs et tout ce qui est nécessaire au bon fonctionnement d'une partie.

				La seule particularité de cette partie réside dans la classe \textbf{GameState} qui comporte des \gls{mutex}, des valeurs atomiques ainsi que des entrées non bloquantes.
				Etant données qu'il y aura de nombreuses entrées/sorties au niveau de cette classe-là, ces différents éléments permettront d'éviter les situations de compétition lors des accès en mémoire de ses attributs.
				
%\newpage
%\section{Index}

\setglossarysection{section}
	\printnoidxglossary[title=Index, style=indexNoDescription]

\end{document}
