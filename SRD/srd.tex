\documentclass{article}

\usepackage[utf8]{inputenc}
\usepackage[T1]{fontenc}
\usepackage{times}
\usepackage[cm]{fullpage}
\usepackage[parfill]{parskip}
\usepackage[xindy]{glossaries}
\usepackage[french]{babel}
\usepackage{lmodern}
\usepackage{graphicx}

\graphicspath{{"Diagrammes UML/"}}

\title{Software Requirement Document : Wizard Poker}
\author{Verhelst Théo \and Reynouard Alexis \and Petit Robin \and Muranovic Allan \and Gueniffey Stanislas \and Berrewaerts Jonathan \and Baudoux Nicolas}

\newglossaryentry{groupe}
{
    name={Groupe},
    description={Explication d'un groupe}
}


\makeglossary

\begin{document}

\pagenumbering{Roman}
\maketitle
\tableofcontents
\newpage
\pagenumbering{arabic}

\section{Introduction}
    \subsection{But du projet}
        L'objectif visé par ce projet est la réalisation d'une application client-serveur en C/C++. L'application visée est un jeu de carte (appelé
        \textit{Wizard Poker}) tour à tour et multijoueurs en réseau. Elle est destinée à tout type de public, \textit{open source}, libre de droit %licence à définir
        et à but non commercial. C'est un projet académique

        Pour ce faire, notre équipe, composée de sept personnes, dispose de trois phases de développement qui dureront approximativement 4 semaines chacune :
        \begin{itemize}
            \item la première portera uniquement sur la création du squelette respectant toutes les demandes effectuées par les client ;
            \item la deuxième phase concernera l'implémentation en console uniquement ;
            \item et enfin la troisième ajoutera l'interface graphique.
        \end{itemize}

    \subsection{Glossaire}  % doit s'agrandir avec le temps
        \printglossaries

    \subsection{Historique}
        \begin{itemize}
            \item[11/12/2015] réunion et squelette du SRD (équipe).
        \end{itemize}

\section{Besoins de l'utilisateur}
    L'utilisateur de l'application, à savoir le joueur, a un certain nombre de besoins. Ces derniers doivent être satisfaits afin de garantir un confort d'utilisation
    maximal. En utilisant l'application, le joueur \textbf{doit} avoir la possibilité de :

    \begin{itemize}
        \item créer un compte utilisateur ;
        \item créer un \gls{deck} ou en modifier un pré-existant ;
        \item consulter les cartes dont il dispose ;
        \item consulter les \glspl{deck} dont il dispose ;
        \item ajouter un joueur existant en ami ;
        \item discuter avec un ami ;
        \item \textbf{\gls{facultatif} :} discuter avec plusieurs amis simultanément ;
        \item affronter un adversaire aléatoire ;
        \item défier un joueur de sa liste d'amis ;
        \item consulter le classement des joueurs.
    \end{itemize}

    \subsection{Exigences fonctionnelles}
        % à exprimer à l'aide de schémas UML
        \begin{center}\includegraphics[scale=0.6]{UseCase1_Main.png}\end{center}

    \subsection{Exigences non-fonctionnelles}
        L'application \textbf{doit} tourner sur Linux (salles 008 et 007 du NO.4) et fonctionner en réseau (\textit{a priori} avec clients
        et serveurs sur des machines séparées).

    \subsection{Exigences de domaine}
        Les exigences implicites au domaine du jeu sont les suivante :

        \begin{itemize}
            \item l'application doit être accessible à tout utilisateur potentiel ;
            \item l'application doit être \textit{amusante} donc équilibrée ;
            \item \textbf{\gls{facultatif} : } empêcher la triche dans la mesure du possible.
        \end{itemize}

\section{Besoins du système}
    \subsection{Exigences fonctionnelles}
        Dans un premier temps, l'application \textbf{doit} tourner en ligne de commande et doit, dans un second temps adopter une interface
        graphique. De plus, l'application doit être développée en C++.

    \subsection{Exigences non-fonctionnelles}
        Le système \textbf{doit} être maintenable et pensé dans l'optique d'une future adaptation avec interface graphique.

    \subsection{Design et fonctionnement du système}

\section{Index}
    % à générer automatiquement avec LaTeX

\end{document}
